
Nature has been and most probably will always be a great conundrum for many that dare try to uncover it's secrets. One of these mysteries is the strong nuclear force. It is one of four fundamental interactions on which the modern physics is built. It might be the most important one for it holds all nuclei and therefore all matter together. 
\newline
Strong interaction has been and will be further studied at the experiment STAR placed at an interaction point of the Relativistic Heavy Ion Collider. It is part of Brookhaven National Laboratory located on Long Island outside of the New York City. Experiment STAR consists of state of the art detectors and calorimeters created to measure particles with cutting edge efficiency and accuracy. Detectors that are crucial for this analysis are the Roman Pots which reconstruct the tracks of forward scattered protons.The focus of this thesis is on diffractive process $p+p \longrightarrow p + X + p$. $X$ represents centrally produced neutral particle $K^0_S$ or $\Lambda^0$ which decays to a hadron pair $\pi^+ \pi^-$ or $p \pi^-$. Antiparticle $\overline{\Lambda}^0$ is also part of the analysis\footnote{Technically, part of $K^0_S$ analysis are it's antiparticles, but because of the decay product, we are not able to differ between them.}. These particles are created through the exchange of two \Pom omerons, a process called the Double \Pom omeron Exchange. A \Pom omeron is part of Regge theory which describes diffractive events very well. The goal for the last several decades has been to describe diffractive events with a broader and more perspective theory called Quantum Chromodynamics. 
\newline
First chapter serves the role of introduction to the area of physics. It involves the categorization of scatterings, kinematic variables that are convenient for description of processes at particle level. This chapter also contains the portrayal of the idea of particle diffraction and Regge theory. At the end of the chapter is a description of the measured process and recent results in this scope of physics. Chapter 2, \autoref{STAR} discusses the collider RHIC, experiment STAR, all the detectors necessary for this thesis and the future of collider physics at Brookhaven National Laboratory. Next chapter begins with explanation of all the conditions imposed on all the measured events part of which is the identification of particles. Lastly, results are displayed and discussed. 