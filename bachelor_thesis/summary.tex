\chapter*{Summary}
Despite its important role in high-energy physics, the Pomeron remains a mysterious object. In this thesis, the Double \Pom omeron exchange is studied and compared to experimental data from diffractive processes measured at experiment STAR, CMS and ISR.
\newline
Chapter 1 begins with introduction to particle physics and to the area of physics which studies diffractive events on top of establishing certain kinematic variables and categorization of scatterings. This chapter ends with discussing relevant recent results in particle diffraction. Chapter 2 continues with a description of the experiment all the way from establishing what Brookhaven National Laboratory and the Relativistic Heavy Ion Collider are to depicting the different detectors installed at the experiment STAR. Crucial role plays the Roman Pot system which registers forward scattered protons and measures their momentum. That enables the STAR collaboration, as the only physics collaboration to do so, to measure diffractive events and differ between exclusive and inclusive events. Chapter 3 characterizes the data sample used for analysis and explains all the various conditions imposed on analyzed events to select those with the most potential. Part of this chapter is particle identification, which is done based on energy loss from the Time Projection Chamber. 
\newline
Finally, chapter 4 discusses the results from the analyzed proton-proton collisions at $\sqrt{s}=510$ GeV data. The focus was on reconstructing particles $K^0_S$ and $\Lambda^0$, which are created through the Double \Pom omeron Exchange mechanism. The reconstructions were done using the major decay channels: $\pi^+ \pi^-$ for $K^0_S$ and $p \pi^-$ for $\Lambda^0$. Peaks for both particles were significant enough to be fitted with Gauss distribution and polynomials to model background. All fitted values for invariant mass of particles $K^0_S$ and $\Lambda^0$ corresponded with results from PDG. Part of the peak for $\Lambda^0$ was it's antiparticle $\overline{\Lambda^0}$ which decays to $\overline{p} \pi^+$. Distributions of invariant mass for $p \pi^-$ and $\overline{p} \pi^+$ were of similar shape but differently scaled. Invariant mass distribution signal for pairs of $p \pi^-$ was $1.3 \pm 0.1(stat)$ larger than for $\overline{p} \pi^+$ pairs. The exclusivity of measured processes was also discussed. For exclusive production, condition on transverse momentum of $\pi^+ \pi^-$ pairs was imposed. The invariant mass distribution showed structures similar to other measurements in this field. Resonances $f_0$(980) and $f_2$(1270) show the presence of something which could possibly be the lightest scalar glueball, but this area of physics needs further study. All the errors included in \autoref{resultS} are statistical. Systematic errors were not calculated but most probably would cause significantly larger uncertainty in the results. 
\newline
Overall, this thesis provides an establishment of the concept of a \Pom omeron, description of the experiment STAR and the analysis of the Double \Pom omeron Exchange in proton-proton collisions at $\sqrt{s}=510$ GeV. The insights gained from this study may broaden knowledge and help in future studies in the area of diffractive events. Aside from that, the work done on this thesis certainly helped in the development of the author.